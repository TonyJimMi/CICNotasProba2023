\noindent Sea A un evento para un experimento con un espacio muestral finito $\Omega$. La probabilidad de $A$ es:

\begin{equation}
    P(A) = \frac{n_{A}}{n_{\Omega}}.
    \label{1.1.2 Probabilidad clásica}
\end{equation}

\noindent Donde $n_{A}$ es el número de casos favorables del evento A y $n_{\Omega}$ es el número de elementos del espacio muestral. \\

\noindent\underline{Nota:} Cada evento del espacio muestral tiene la misma probabilidad de ocurrir.\\

\noindent\textbf{Ejemplo: } Si se desea calcular la probabilidad de que al tirar un dado salga un número par, el espacio muestral sería 

\begin{equation*}
    \Omega = \{1,2,3,4,5,6\},
\end{equation*}

\noindent por lo que $n_{\Omega} = 6$ y $n_{A = \{2,4,6\}} = 3$, por lo tanto

\begin{equation*}
    P(A) = \frac{3}{6} = \frac{1}{2}
\end{equation*}