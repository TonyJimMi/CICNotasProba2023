\noindent La probabilidad frecuentista se refiere a qué tan probable resulta un suceso si un experimento se repite muchas veces.

\begin{equation}
    P(A) = \frac{n(A)}{n}
    \label{1.1.2 Probabilidad frecuentista}
\end{equation}

\noindent\textbf{Ejemplo: } Supongamos que se ha lanzado un moneda diez veces, de tal manera que en tres ocasiones ha salido águila, y siete veces ha salido sol. Entonces la probabilidad frecuentista de que haya salido ágila es

\begin{equation*}
    P(A_{\text{Águila}}) = \frac{3}{10}
\end{equation*}

\noindent mientras que la probabilidad para el caso en el que salió sol es

\begin{equation*}
    P(A_{\text{Sol}}) = \frac{7}{10}.
\end{equation*}