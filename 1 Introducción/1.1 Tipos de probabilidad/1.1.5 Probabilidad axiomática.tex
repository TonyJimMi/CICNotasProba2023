\noindent Dado un experimento y un espacio muestral $\Omega$, se asigna a cada evento $A$ un número $P(A)$, que debe satisfacer los axiomas siguientes: \\

\noindent\textbf{Axioma 1:} Para cualquier evento $A$
\begin{equation*}
    0 \leq P(A)
\end{equation*}

\noindent\textbf{Axioma 2:} 
\begin{equation*}
    P(\Omega) = 1
\end{equation*}

\noindent\textbf{Axioma 3:} Si $A_{1}, A_{2}, ...$ es un conjunto de eventos mutuamente excluyentes, entonces:
\begin{equation*}
    P(A_{1} \cup A_{2} \cup ...) = \sum_{i = 1}^{\infty} P(A_{i})
\end{equation*}


