\noindent \textbf{Definici\'on.} \\
    Un conjunto puede definirse como una lista o colección bien definida de objetos; los objetos comprendidos en un conjunto son llamados sus ‘elementos o miembros’. \\

\noindent    \textbf{Ejemplos:}
    \begin{itemize}
        \item Los n\'umeros $1, 3, 7$ y $10$,
        \item Los estudiantes del CIC, 
        \item Los r\'ios de M\'exico,
        \item Los n\'umeros pares enteros: $0, 2, 4, 6...$
    \end{itemize}

\noindent    \textbf{Notaci\'on.} \\ 
\noindent     Un conjunto puede ser denotado con letras may\'usculas, como por ejemplo,
     \begin{center}
         $A, B, C, X, Y, ...$
     \end{center}
 
\noindent     Mientras que las letras min\'usculas se utilizan para indicar sus elementos.
     \begin{center}
         $a,b,c,x,y,z, ...$
     \end{center}

\noindent  Existen dos formas de especificar un conjunto: \\ 
    
    \begin{itemize}{}
        \item  \textbf{M\'etodo de extensi\'on:} Que consiste en listar sus elementos.
        \item \textbf{M\'etodo de comprensi\'on:} Describir alguna propiedad conservada por todos los miembros del conjunto.\\ 
    \end{itemize}

   
    \begin{center}
        \begin{tcolorbox}
        \textbf{Ejemplo: } El conjunto de las vocales del alfabeto.      \vspace{5px}
            \begin{itemize}
                \item $ V = \lbrace a, e, i, o, u \rbrace $ \hspace{2.35cm} \textbf{M\'etodo de extensi\'on } 
                \item $ V =\lbrace x \mid x $ es una vocal $\rbrace$  \hspace{1cm} \textbf{M\'etodo de comprensi\'on} \\ \vspace{2px}
                 Léase \textit{“El conjunto de los elementos x tal que x es una vocal”.}
            \end{itemize}
        \end{tcolorbox} 
    \end{center}