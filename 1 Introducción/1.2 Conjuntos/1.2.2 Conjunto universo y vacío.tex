
\noindent   \textbf{Conjunto universo.} \\
\noindent     Todos los conjuntos están contenidos en algún gran conjunto fijo, llamado \textbf{conjunto universal o universo}. Denotado por:
    \begin{center}
        $U$,
    \end{center}
\noindent     a menos que se especifique lo contrario.\\ \vspace{6px}
    
\noindent \textbf{Conjunto vac\'io.} \\
\noindent     Tambi\'en llamado \textbf{conjunto nulo}. Es aquel que no contiene elementos en él, se denota como, 
    \begin{center}
        $\emptyset$ \hspace{0.5cm} \'o \hspace{0.5cm} $\lbrace \rbrace$.
    \end{center}
\noindent     \'Este es subconjunto de cualquier otro conjunto. 