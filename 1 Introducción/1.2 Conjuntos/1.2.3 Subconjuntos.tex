\noindent Si cada elemento de un conjunto $A$ es también un elemento de un conjunto $B$, entonces $A$ se le llama \textbf{subconjunto} de $B$. También decimos que $A$ está contenido en $B$ o $B$ contiene a $A$. Esta relación se escribe:

    \begin{equation}
            A \subseteq B  \hspace{0.5cm} \text{\'o} \hspace{0.5cm} B \supseteq A
    \end{equation}

\noindent     si $A$ no es un subconjunto de $B$, es decir, si al menos un elemento de $A$ no pertenece a $B$, escribimos:

    \begin{equation}
            A \nsubseteq B  \hspace{0.5cm}  \text{\'o} \hspace{0.5cm} B \nsupseteq A
    \end{equation}


\noindent \textbf{Conjutos disjuntos.} \\
\noindent     Dos conjuntos $A$ y $B$ son disjuntos si no tienen ning\'un elemento en com\'un, tales que  $A \cap B = \emptyset$. \\
\noindent \textbf{Por ejemplo:}
    \begin{center}
         $A = \lbrace 1, 3\rbrace $\hspace{0.5cm}  $B = \lbrace 1, 2, 3\rbrace$  \hspace{0.5cm}  $C = \lbrace  2\rbrace$
    \end{center}
   
    $A$ y $C$ son conjuntos disjuntos.