
    
 \noindent   Formas de representar una relación $R$ de $A$ a $B$.  \\
 
 \noindent \textbf{Por ejemplo,} si $R$ es una relación de $A = \{1, 2, 3\}$ a $B = \{x, y, z\}$, y $R = \{(1, y), (1, z), (3, y)\}$. \\
    

    \noindent \textbf{ Matriz de la relación.}\\
    \noindent         Formar un arreglo rectangular cuyas filas estén etiquetadas por los elementos de $A$ y cuyas columnas estén etiquetadas por los elementos de $B$. Poner un $1$ o $0$ en cada posición del arreglo según si $a \in A$ está o no relacionada con $b \in B$.

                
                \[
                \begin{bmatrix}
                    & x & y & z \\
                  1 & 0 & 1 & 1 \\
                  2 & 0 & 0 & 0 \\
                  3 & 0 & 1 & 0 \\
                \end{bmatrix}
                \]

    \vspace{10px}
    \noindent \textbf{Diagrama de flechas de la relación}.\\
             Escribir los elementos de $A$ y los elementos de $B$ en dos conjuntos disjuntos, y luego dibujar una flecha desde $a \in A$ hasta $b \in B$ siempre que $a$ esté relacionado con $b$. 