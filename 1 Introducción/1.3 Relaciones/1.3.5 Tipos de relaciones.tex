 \noindent  Considere el conjunto $A$ dado.

 \begin{itemize}
    \item \textbf{Relaciones Reflexivas}: Una relación $R$ en el conjunto $A$ es reflexiva si $a$ $R$ $a$ para cada $a \in A$, es decir, si $(a, a) \in R$ donde $a \in A$. Por tanto, $R$ no es reflexiva si existe un $a \in A$ tal que $(a, a) \notin R$.\\


    \item  \textbf{Relaciones Simétricas}: Una relación $R$ en un conjunto $A$ es simétrica si siempre que $a$ $R$ $b$, entonces $b$ $R$ $a$, es decir, si siempre que $(a, b) \in R$, entonces $(b, a) \in R$. Por tanto, $R$ no es simétrica si existe $a, b \in A$ tal que $(a, b) \in R$ pero $(b, a) \notin R$. \\

    \item \textbf{Relaciones Antisimétricas}: Una relación $R$ en un conjunto $A$ es antisimétrica si siempre que $a$ $R$ $b$ y $b$ $R$ $a$, entonces $a = b$, es decir, si siempre que $(a, b)$ y $(b, a)$ pertenecen a $R$, entonces $a = b$. Por tanto, $R$ no es antisimétrica si existe $a, b \in A$ tal que $(a, b)$ y $(b, a)$ pertenecen a $R$, pero $a$ es diferente de $b$. \\

    \item    \textbf{Relaciones Transitivas}: Una relación $R$ en un conjunto $A$ es transitiva si siempre que $a$ $R$ $b$ y $b$ $R$ $c$, entonces $a$ $R$ $c$, es decir, si siempre que $(a, b), (b, c) \in R$, entonces $(a, c) \in R$. Así que $R$ no es transitiva si existe $a, b, c \in A$ tal que $(a, b), (b, c) \in R$, pero $(a, c) \notin R$. \\
\end{itemize}