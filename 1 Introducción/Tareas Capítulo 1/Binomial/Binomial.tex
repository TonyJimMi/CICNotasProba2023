%---------------------------------------------
%                Ejercicio 1
%---------------------------------------------

\noindent \textbf{Ejercicio 1:} Demostrar que

\begin{equation*}
    \binom{n}{k} = \binom{n}{n-k}
\end{equation*}

\noindent \textbf{Solución: } Recordemos que

\begin{equation}
    \binom{n}{k} = \frac{n!}{(n-k)! k!}
    \label{C1: Tarea Binomio de Newton 1}
\end{equation}

\noindent por otro lado

\begin{equation}
    \begin{split}
        \binom{n}{n-k} & = \frac{n!}{(n-(n-k))! (n-k)!} \\
        & = \frac{n!}{k!(n-k)!}
    \end{split}
    \label{C1: Tarea Binomio de Newton 2}
\end{equation}

\noindent Por lo tanto, por (\ref{C1: Tarea Binomio de Newton 1}) y (\ref{C1: Tarea Binomio de Newton 2}):

\begin{equation}
    \binom{n}{k} = \binom{n}{n-k}
\end{equation}

%---------------------------------------------
%                Ejercicio 2
%---------------------------------------------

\noindent \textbf{Ejercicio 2:} Demostrar que

\begin{itemize}
    \item [a) ]
    \begin{equation*}
        \binom{n}{0} = 1
    \end{equation*}

    \item[b) ] 
    \begin{equation*}
        \binom{0}{k} = 0
    \end{equation*}
\end{itemize}

\noindent \textbf{Solución: } 

\begin{itemize}
    \item [a) ]

    \begin{equation*}
        \begin{split}
            \binom{n}{0} & = \frac{n!}{(n)! 0!} \\
            & = \frac{n!}{n!(1)} \\
            & = 1
        \end{split}
    \end{equation*}

    \item [b) ] La expresión $\binom{n}{k}$ se define como el número de formas en que se pueden elegir $k$ elementos de un conjunto de $n$ elementos, pero si $n=0$, no hay elementos en el conjunto, por lo tanto, $\binom{0}{k}$ debe ser igual a cero.
\end{itemize}

