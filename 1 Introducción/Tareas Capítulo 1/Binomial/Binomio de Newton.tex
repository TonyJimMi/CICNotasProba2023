\noindent \textbf{Binomio de Nexton}.\\
    \noindent Un \textbf{binomio} es un polinomio que consta de dor t\'erminos, como: 
    \begin{center}
        $a+b$, $x^{2}-y$, $\frac{a}{3}-\frac{5mx^{4}}{6b^{2}}$...
    \end{center}
    
    \noindent Existe un procedimiento  para calcular la potencia de un binomio, llamado \textbf{\textit{teorema del binomio de Newton}} o fórmula para el binomio de Newton. Donde para un n\'umero $n$ el desarrollo de:

    \begin{align*}
(a + b)^n &= \binom{n}{0}a^n b^0 + \binom{n}{1}a^{n-1}b^1 + \binom{n}{2}a^{n-2}b^2 + \ldots + \binom{n}{n-1}a^1b^{n-1} + \binom{n}{n}a^0b^n \\
&= a^n + \binom{n}{1}a^{n-1}b + \binom{n}{2}a^{n-2}b^2 + \ldots + \binom{n}{n-1}ab^{n-1} + b^n
\end{align*}\\

    Queda simplificado en:
    \begin{equation}
        (a + b)^n = \sum_{k=0}^{n} \binom{n}{k} a^{n-k} b^k
    \end{equation}\\

\noindent \textit{\textbf{Donde Si $n$ es antural, el desarrollo de $(a+b)^{n}$ cumple con las siguientes caracter\'isticas:}}

\begin{itemize}{}
    \item El primer t\'ermino es $a^{n}$ y el \'ultimo t\'ermino es $b^{n}$.
    \item Al desarrollar el binomio se obitnen $(n+1)$ t\'erminos.
    \item Conforme aumentan los t\'erminos, la potencia del primer t\'ermino $a$ disminuye en 1 y la del segundo t\'ermino aumenta en 1.
    \item Para obtener el i-\'esimo t\'ermino se utiliza la f\'ormula:


    \begin{equation}
        \text{i-\'esimo} = \binom{n}{i} \cdot a^{n-i} \cdot b^i
    \end{equation}
    
    \item Sus coeficientes llamados \textit{coeficientes binomiales}, están dados por la f\'ormula:

    \begin{equation}
        \binom{n}{k} = \frac{n!}{(k!(n-k)!} = \frac{n(n-1)(n-2)\dotsb(n-k+1)}{k!} \\
    \end{equation}

    siendo $n!$ el \textit{factorial} de un n\'umero $n$ natural, el cual se define como

    \begin{center}
         $n! = \begin{cases}
                      1 & \text{si } n = 0 \\
                      n \cdot (n-1)! & \text{si } n \geq 1
                \end{cases}$
    \end{center}

  


\end{itemize}

    
    
\noindent Primeros 10 ejemplos del desarollo del Binomio de Newton: \\

\begin{align*}
    (x + y)^0 & = 1\\  (x + y)^1 & = x + y \\
    (x + y)^2 & = x^2 + 2xy + y^2 \\
    (x + y)^3 & = x^3 + 3x^2y + 3xy^2 + y^3 \\
    (x + y)^4 & = x^4 + 4x^3y + 6x^2y^2 + 4xy^3 + y^4 \\
    (x + y)^5 & = x^5 + 5x^4y + 10x^3y^2 + 10x^2y^3 + 5xy^4 + y^5 \\
    (x + y)^6 & = x^6 + 6x^5y + 15x^4y^2 + 20x^3y^3 + 15x^2y^4 + 6xy^5 + y^6 \\
    (x + y)^7 & = x^7 + 7x^6y + 21x^5y^2 + 35x^4y^3 + 35x^3y^4 + 21x^2y^5 + 7xy^6 + y^7 \\
    (x + y)^8 & = x^8 + 8x^7y + 28x^6y^2 + 56x^5y^3 + 70x^4y^4 + 56x^3y^5 + 28x^2y^6 + 8xy^7 + y^8 \\
    (x + y)^9 & = x^9 + 9x^8y + 36x^7y^2 + 84x^6y^3 + 126x^5y^4 + 126x^4y^5 + 84x^3y^6 + 36x^2y^7 + 9xy^8 + y^9 \\
    (x + y)^{10} & = x^{10} + 10x^9y + 45x^8y^2 + 120x^7y^3 + 210x^6y^4 + 252x^5y^5 + 210x^4y^6 + 120x^3y^7 + 45x^2y^8 + 10xy^9 + y^{10}
\end{align*}   
