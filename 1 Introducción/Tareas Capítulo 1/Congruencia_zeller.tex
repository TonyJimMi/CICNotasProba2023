\noindent La Congruencia de Zeller es una fórmula matemática que se utiliza para calcular el día de la semana correspondiente a una fecha específica. Esta fórmula fue desarrollada por Christian Zeller en el siglo XIX. La fórmula toma en cuenta el día, el mes y el año de la fecha y devuelve un número que representa el día de la semana, donde 0 representa el sábado, 1 el domingo, 2 el lunes, y así sucesivamente.

\noindent La fórmula de congruencia de Zeller se define de la siguiente manera:

\begin{equation*}
	h = (q + 13*(m+1)/5 + K + K/4 + J/4 + 5*J) \; mod \; 7
\end{equation*}


\noindent Donde:

\begin{itemize}
	\item h es el día de la semana (0 = sábado, 1 = domingo, 2 = lunes, ..., 6 = viernes).
	
	\item q es el día del mes.
	
	\item m es el mes (3 = marzo, 4 = abril, ..., 12 = diciembre, enero y febrero se cuentan como 13 y 14 del año anterior).
	
	\item K es el año del siglo (el año sin los dos dígitos finales).
	
	\item J es el siglo (los dos primeros dígitos del año).
\end{itemize}

\begin{lstlisting}
	def zeller_congruence(day, month, year):
		# Ajuste del mes y el año para la fórmula de Zeller
		if month in (1, 2):
			month += 12
			year -= 1
	
		# Aplicación de la fórmula de Zeller
		K = year % 100
		J = year // 100
		day_of_week = (day + 13 * (month + 1) // 5 + K + K // 4 + J // 4 - 2 * J) % 7
	
		# Días de la semana
		days = ["Sábado", "Domingo", "Lunes", "Martes", "Miércoles", "Jueves", "Viernes"]
	
		return days[day_of_week]
	
	# Ingresa la fecha en formato DD, MM, YYYY
	day = int(input("Ingresa el día (1-31): "))
	month = int(input("Ingresa el mes (1-12): "))
	year = int(input("Ingresa el año (YYYY): "))
	
	day_of_week = zeller_congruence(day, month, year)
	print(f"El día de la semana es: {day_of_week}")
	
	
\end{lstlisting}