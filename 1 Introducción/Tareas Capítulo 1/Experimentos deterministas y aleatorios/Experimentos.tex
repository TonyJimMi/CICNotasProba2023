\noindent \textbf{Experimento determinista}: Un experimento determinista es aquel que tiene un resultado predecible y reproducible siempre que se mantengan las mismas condiciones y parámetros iniciales. \\

\noindent \textbf{Ejemplos de expermientos deterministas}

\begin{itemize}
    \item Medir la longitud de una mesa.

    \item Calcular el área de un círculo.

    \item Sumar de dos números enteros

    \item Encender una bombilla con el interruptor

    \item Hervir agua a $100 ^{o}C$ 

    \item Medir la velocidad de una pelota deslizándose por una pendiente desde la misma altura.

    \item Medir la velocidad al disparar una bala.

    \item Medir la velocidad de un objeto que cae desde una ventana.

    \item Juntar un imán con un metal.

    \item Medir la corriente eléctrica de un circuito.
\end{itemize}

\noindent \textbf{Experimento aleatorio}: Un experimento aleatorio es aquel en el que si lo repetimos con las mismas condiciones iniciales no garantiza los mismos resultados. \\

\noindent \textbf{Ejemplos de expermientos aleatorios}

\begin{itemize}
    \item Lanzar una moneda y observar si sale cara o cruz. 

    \item Extraer una carta de una baraja y ver su palo y número. 

    \item Tirar un dado y ver el número que aparece en la cara superior. 

    \item Elegir una persona al azar de un grupo y preguntarle su edad. 

    \item Comprar un boleto de lotería y ver si resulta premiado.

    \item Medir el spin de un electrón.

    \item Escoger una bola de un saco.

    \item Elegir las fichas de domino.

    \item Apostar a un número en la ruleta.

    \item Desarmar un cubo de rubik sin un orden.
\end{itemize}