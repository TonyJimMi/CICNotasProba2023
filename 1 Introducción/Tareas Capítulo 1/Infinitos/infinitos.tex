\noindent \textbf{Infinitos:} 1. Infinito Contable (\( \aleph_0 \) o Aleph-Cero): Este es el tipo de infinito que se asocia con conjuntos infinitos que tienen una correspondencia uno a uno con los números naturales (\( 1, 2, 3, \ldots \)). Por ejemplo, el conjunto de todos los números naturales es infinito contable.\\

\noindent 2. Infinito No Contable (\( \aleph_1 \) o Aleph-Uno): Estos son conjuntos infinitos que no pueden ponerse en correspondencia uno a uno con los números naturales. El conjunto de todos los números reales es un ejemplo de conjunto infinito no contable. \\

\noindent 3. Infinito más grande (Infinito no enumerable): Cantor demostró que hay infinitos de diferentes "tamaños" y que no podemos enumerar todos los conjuntos infinitos. Por lo tanto, existe un infinito más grande que el infinito contable y el infinito no contable, y este infinito se llama "infinito no enumerable". \\

\noindent 4. Infinito Ordinal: En teoría de conjuntos, se definen números ordinales que representan ciertos tipos de orden entre conjuntos infinitos. Estos ordinales son diferentes de los números cardinales que describen el tamaño de los conjuntos infinitos. Por ejemplo, \( \omega \) (omega) es el primer ordinal infinito y representa la ordenación de los números naturales. \\
