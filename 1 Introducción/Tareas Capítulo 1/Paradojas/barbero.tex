\noindent \textbf{Paradojas y antinomias:} 1. Paradoja: Una paradoja es una declaración o situación que parece llevar a una contradicción lógica o a resultados inesperados y aparentemente contradictorios. \\

\noindent 2. Antinomia: El término "antinomia" no se utiliza comúnmente en el mismo sentido que "paradoja". Sin embargo, en algunos contextos, una antinomia se refiere a una contradicción entre dos afirmaciones o principios que, en teoría, deberían ser mutuamente excluyentes. Una antinomia puede surgir cuando dos reglas o axiomas se contradicen entre sí, lo que plantea un desafío a la consistencia lógica de un sistema o conjunto de creencias. En general, una antinomia se considera un tipo específico de paradoja que involucra principios o reglas en conflicto. \\

\noindent En resumen, una paradoja es una situación que parece llevar a una contradicción lógica o a resultados sorprendentes, mientras que una antinomia es una contradicción entre dos principios o reglas que deberían ser mutuamente excluyentes. Ambos términos se utilizan en el contexto de la lógica, la filosofía y otros campos para describir situaciones que desafían nuestras expectativas racionales. \\

\noindent 1. Paradoja de Russell (también conocida como Paradoja del Conjunto de Todos los Conjuntos que no se Contienen a Sí Mismos): \\

\noindent La Paradoja de Russell se presenta cuando consideramos el conjunto que contiene todos los conjuntos que no se contienen a sí mismos. La pregunta es si este conjunto se contiene a sí mismo o no. Si asumimos que se contiene a sí mismo, entonces no debería incluirse en el conjunto, ya que solo debería contener conjuntos que no se contienen a sí mismos. Por otro lado, si asumimos que no se contiene a sí mismo, entonces debería incluirse en el conjunto, ya que cumple con la condición de no contenerse a sí mismo. Esta paradoja muestra una contradicción en la teoría de conjuntos ingenua y condujo al desarrollo de la teoría de conjuntos axiomática para evitar paradojas como esta. \\

\noindent 2. Antinomia de Cantor: Aunque el término "antinomia" no se usa comúnmente en este contexto, se puede considerar la paradoja de Cantor como un ejemplo de antinomia en la teoría de conjuntos. Georg Cantor demostró que no se puede establecer una correspondencia uno a uno entre el conjunto de todos los números naturales y el conjunto de todos los números reales, lo que implica que hay infinitos de diferentes "tamaños". Sin embargo, esto parece contradecir la intuición de que todos los infinitos son iguales en tamaño. Esta aparente contradicción entre la intuición y el resultado matemático se puede considerar una antinomia en el sentido de que desafía nuestras expectativas comunes. \\

\noindent En resumen, una paradoja es una declaración o situación que parece llevar a una contradicción lógica o a resultados inesperados, como la Paradoja de Russell en la teoría de conjuntos. Por otro lado, una antinomia podría considerarse una contradicción aparente o un conflicto entre la intuición y los resultados matemáticos, como la Antinomia de Cantor en el contexto de los infinitos en la teoría de conjuntos. Ambas destacan la importancia de una formulación cuidadosa de las teorías matemáticas y lógicas. \\

\noindent \textbf{Paradoja del barbero:} En un lejano poblado de un antiguo emirato había un barbero llamado As-Samet diestro en afeitar cabezas y barbas, maestro en escamondar pies y en poner sanguijuelas. Un día el emir se dio cuenta de la falta de barberos en el emirato, y ordenó que los barberos solo afeitaran a aquellas personas que no pudieran afeitarse a sí mismos. ¡Ah! e impuso la norma de que todo el mundo estuviera afeitado, (no se sabe si por higiene, por estética, o por demostrar que podía imponer su santa voluntad y mostrar así su poder). Cierto día el emir llamó a As-Samet para que lo afeitara y él le contó sus angustias:\cite{Paradoja_del_barbero} \\

 \noindent —En mi pueblo soy el único barbero. No puedo afeitar al barbero de mi pueblo, ¡que soy yo!, ya que si lo hago, entonces puedo afeitarme por mí mismo, por lo tanto ¡no debería afeitarme! pues desobedecería vuestra orden. Pero, si por el contrario no me afeito, entonces algún barbero debería afeitarme, ¡pero como yo soy el único barbero de allí!, no puedo hacerlo y también así desobedecería a vos mi señor, oh emir de los creyentes, ¡que Allah os tenga en su gloria!\\

\noindent El emir pensó que sus pensamientos eran tan profundos, que lo premió con la mano de la más hermosa de sus concubinas. Así, el barbero As-Samet vivió para siempre feliz y barbón.
