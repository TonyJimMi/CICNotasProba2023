
\noindent \textbf{Definición del Problema:}

\noindent Supongamos que tenemos un hotel con un número infinito de habitaciones, numeradas desde $1$ hasta $n$ (donde $n$ es infinito). Además, imaginemos que todas las habitaciones están ocupadas.\\

\noindent Ahora, llega un nuevo huésped al hotel, y nos preguntamos: ¿Podemos acomodar a este nuevo huésped sin rechazar a ninguno de los huéspedes actuales?\\

\noindent \textbf{Solución Sorprendente:} \\

\noindent La solución propuesta por Georg Cantor, el matemático que introdujo esta paradoja, es sorprendente. En lugar de simplemente decir que no hay espacio, Cantor sugirió que podríamos mover a cada huésped actual a la habitación siguiente.\\

\noindent Por ejemplo, el huésped en la habitación $1$ se mueve a la habitación $2$, el huésped en la habitación $2$ se mueve a la habitación $3$, y así sucesivamente. Este proceso se realiza para todas las habitaciones ocupadas. Luego, la habitación $1$ queda libre para el nuevo huésped.\\

\noindent Dado que estamos tratando con un conjunto infinito de habitaciones, siempre habrá una "habitación siguiente" para cada huésped, y podemos hacer espacio para un nuevo huésped de esta manera. \\

\noindent \textbf{Implicaciones Filosóficas y Matemáticas:} \\

\noindent La Paradoja del Hotel Infinito plantea cuestiones profundas sobre la naturaleza del infinito. En un conjunto infinito, las reglas de intuición cotidiana pueden no aplicarse de la misma manera que lo hacen en conjuntos finitos. \\

\noindent Esta paradoja fue un componente clave en el trabajo de Cantor sobre la teoría de conjuntos y los números transfinitos. Cantor demostró que hay diferentes "tamaños" de infinito, y esto cambió fundamentalmente la forma en que los matemáticos piensan sobre el concepto de infinito.\\

\noindent En resumen, la Paradoja del Hotel Infinito es un ejemplo intrigante de cómo los conceptos abstractos en matemáticas, como el infinito, pueden llevar a resultados sorprendentes que desafían nuestra intuición inicial. Además, esta paradoja ha influido en el desarrollo de la teoría de conjuntos y ha tenido implicaciones filosóficas significativas en nuestra comprensión del infinito. \\
