%\noindent \textbf{Ant Odometer}
\section*{Ant odometer}

\noindent Las hormigas son insectos sociales que evolucionaron a partir de avispas vespoideas a mediados del período Cretácico, hace unos 150 millones de años. Tras el surgimiento de las plantas con flores, hace unos 100 millones de años, las hormigas se diversificaron en numerosas especies.\\

\noindent La hormiga del desierto del Sahara, Cataglyphis fortis, recorre inmensas distancias sobre terreno arenoso, a menudo completamente desprovisto de puntos de referencia, en busca de alimento. Estas criaturas pueden regresar a su nido utilizando una ruta directa en lugar de volver sobre su camino de salida. No sólo juzgan direcciones, usando la luz del cielo como orientación, sino que también parecen tener una "computadora" incorporada que funciona como un podómetro que cuenta sus pasos y les permite medir distancias exactas. Una hormiga puede viajar hasta 50 metros hasta que encuentra un insecto muerto, después de lo cual arranca un trozo para llevarlo directamente a su nido, al que se accede a través de un agujero que a menudo tiene menos de un milímetro de diámetro.\\

\noindent Al manipular la longitud de las patas de las hormigas para darles zancadas más largas y más cortas, un equipo de investigación de científicos alemanes y suizos descubrió que las hormigas "cuentan" los pasos para juzgar la distancia. Por ejemplo, una vez que las hormigas llegaron a su destino, las patas se alargaron añadiendo zancos o se acortaron mediante una amputación parcial. Luego, los investigadores devolvieron las hormigas para que pudieran comenzar su viaje de regreso al nido. Las hormigas con zancos viajaron demasiado lejos y pasaron la entrada del nido, mientras que las que tenían las patas amputadas no llegaron. Sin embargo, si las hormigas comenzaron su viaje desde su nido con las patas modificadas, pudieron calcular las distancias apropiadas. Esto sugiere que la longitud de la zancada es el factor crucial. Además, el cerebro de la hormiga le permite calcular una cantidad relacionada con la proyección horizontal de su camino para que no se pierda incluso si el paisaje arenoso cambia durante su viaje.