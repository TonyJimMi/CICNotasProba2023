\section*{Los primates cuentan}

\noindent Hace alrededor de 60 millones de años, pequeños primates parecidos a lémures habían evolucionado en diversas partes del mundo, y hace 30 millones de años ya existían primates con características similares a los monos. ¿Podían contar tales criaturas? El significado del conteo por parte de los animales es un tema altamente controvertido entre los expertos en comportamiento animal. Sin embargo, muchos académicos sugieren que los animales tienen algún sentido del número. H. Kalmus escribe en su artículo "Animals as Mathematicians" publicado en la revista Nature: \\

\begin{quote}
  No hay duda de que algunos animales, como las ardillas o los loros, pueden ser entrenados para contar. ... Se han informado facultades de conteo en ardillas, ratas y para insectos polinizadores. Algunos de estos animales y otros pueden distinguir números en patrones visuales de lo contrario similares, mientras que otros pueden ser entrenados para reconocer e incluso reproducir secuencias de señales acústicas. Incluso algunos pueden ser entrenados para golpear el número de elementos (puntos) en un patrón visual.... La falta del numeral hablado y el símbolo escrito hace que a muchas personas les cueste aceptar a los animales como matemáticos.
\end{quote}

\noindent Se ha demostrado que las ratas "cuentan" al realizar una actividad el número correcto de veces a cambio de una recompensa. Los chimpancés pueden presionar números en una computadora que coinciden con la cantidad de plátanos en una caja. Testsuro Matsuzawa, del Instituto de Investigación de Primates de la Universidad de Kyoto en Japón, enseñó a un chimpancé a identificar números del 1 al 6 presionando la tecla de la computadora correspondiente cuando se le mostraba una cierta cantidad de objetos en la pantalla de la computadora. \\

\noindent Michael Beran, científico investigador en la Universidad Estatal de Georgia en Atlanta, Georgia, entrenó a chimpancés para usar una pantalla de computadora y un joystick. La pantalla mostraba un número y luego una serie de puntos, y los chimpancés tenían que hacer coincidir ambos. Un chimpancé aprendió los números del 1 al 7, mientras que otro logró contar hasta 6. Cuando los chimpancés fueron probados nuevamente después de tres años, ambos pudieron igualar los números, pero con el doble de la tasa de error.
