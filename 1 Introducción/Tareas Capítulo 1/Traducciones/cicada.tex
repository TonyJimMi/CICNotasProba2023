
\section*{Números primos generados por cígarras}

\noindent Las cícadas son insectos alados que evolucionaron alrededor de hace 1.8 millones de años durante el Pleistoceno, cuando los glaciares avanzaron y retrocedieron a lo largo de América del Norte. Las cícadas del género Magicicada pasan la mayor parte de sus vidas bajo tierra, alimentándose de los jugos de las raíces de las plantas, para luego emerger, aparearse y morir rápidamente. \\ 

\noindent Estas criaturas muestran un comportamiento sorprendente: su emergencia está sincronizada con períodos de años que suelen ser números primos, como 13 y 17. (Un número primo es un entero como 11, 13 y 17 que tiene solo dos divisores enteros: 1 y él mismo). Durante la primavera de su año 13º o 17º, estas cícadas periódicas construyen un túnel de salida. \\

\noindent A veces, más de 1.5 millones de individuos emergen en un solo acre; esta abundancia de cuerpos puede tener valor de supervivencia al abrumar a depredadores como pájaros que no pueden posiblemente comerlos todos de una vez.
Algunos investigadores han especulado que la evolución de ciclos de vida de números primos ocurrió para que las criaturas aumentaran sus posibilidades de evadir a depredadores y parásitos de vida más corta. Por ejemplo, si estas cícadas tuvieran ciclos de vida de 12 años, todos los depredadores con ciclos de vida de 2, 3, 4 o 6 años podrían encontrar más fácilmente a los insectos. Mario Markus del Instituto Max Planck de Fisiología Molecular en Dortmund, Alemania, y sus colegas descubrieron que este tipo de ciclos de números primos surgen naturalmente a partir de modelos matemáticos evolutivos de interacciones entre depredador y presa. Para experimentar, asignaron primero duraciones de ciclo de vida aleatorias a sus poblaciones simuladas por computadora. Después de algún tiempo, una secuencia de mutaciones siempre bloqueaba a las cícadas sintéticas en un ciclo de números primos estable. \\

