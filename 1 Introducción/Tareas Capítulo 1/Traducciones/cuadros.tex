\section*{Cuadrados mágicos}

\noindent Se sugiere que los cuadrados mágicos se originaron en China y fueron mencionados por primera vez en un manuscrito de la época del Emperador Yu, alrededor del 2200 a.C. Un cuadrado mágico consta de $N^2$ cajas, llamadas celdas, llenas de enteros que son todos diferentes. Las sumas de los números en las filas horizontales, columnas verticales y diagonales principales son todas iguales. \\

\noindent Si los enteros en un cuadrado mágico son los números consecutivos del 1 al $N^2$, se dice que el cuadrado es de orden $N$, y el número mágico, o suma de cada fila, es una constante igual a $\frac{N(N^2 + 1)}{2}$. El artista renacentista Albrecht Dürer creó este maravilloso cuadrado mágico de $4 \times 4$ en 1514. \\

\[
\begin{array}{|c|c|c|c|}
\hline
16 & 13 & 3 & 2 \\
\hline
5 & 10 & 11 & 8 \\
\hline
9 & 6 & 7 & 12 \\
\hline
4 & 15 & 14 & 1 \\
\hline
\end{array}
\]

\noindent Observa que los dos números centrales en la fila inferior leen "1514", el año de su construcción. Las filas, columnas y diagonales principales suman 34. Además, 34 es la suma de los números de las esquinas (16 + 13 + 4 + 1) y del cuadrado central $2 \times 2$ (10 + 11 + 6 + 7). \\

\noindent Tan temprano como en 1693, los 880 diferentes cuadrados mágicos de cuarto orden fueron publicados póstumamente en \textit{Des quassez ou tables magiques} por Bernard Frenicle de Bessy, un destacado aficionado matemático francés y uno de los principales investigadores de cuadrados mágicos de todos los tiempos. \\

\noindent Hemos llegado muy lejos desde los más simples cuadrados mágicos de $3 \times 3$ venerados por civilizaciones de casi todos los períodos y continentes, desde los indios mayas hasta el pueblo Hasua de África. Hoy en día, los matemáticos estudian estos objetos mágicos en dimensiones altas, por ejemplo, en forma de hipercubos de cuatro dimensiones que tienen sumas mágicas en todas las direcciones apropiadas.
