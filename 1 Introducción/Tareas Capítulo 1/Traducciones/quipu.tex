\section*{Quipu}

\noindent Los antiguos incas usaban quipus, bancos de memoria hechos de cuerdas y nudos, para almacenar números. Hasta hace poco, los quipus más antiguos conocidos databan alrededor del año 650 d.C. Sin embargo, en 2005, un quipu de la ciudad costera peruana de Caral fue fechado hace aproximadamente 5,000 años.\\

\noindent Los incas de América del Sur tenían una civilización compleja con una religión estatal común y un idioma común. Aunque no tenían escritura, mantenían registros extensos codificados por un sistema lógico-numérico en los quipus, que variaban en complejidad desde tres hasta alrededor de mil cuerdas. Desafortunadamente, cuando los españoles llegaron a América del Sur, vieron los extraños quipus y pensaron que eran obras del diablo. Los españoles destruyeron miles de ellos en nombre de Dios, y hoy en día solo quedan alrededor de 600 quipus. \\

\noindent Los tipos y posiciones de nudos, las direcciones de las cuerdas, los niveles de las cuerdas, y el color y el espaciado representan números mapeados a objetos del mundo real. Se utilizaron diferentes grupos de nudos para diferentes potencias de 10. Probablemente, los nudos se utilizaron para registrar recursos humanos y materiales e información del calendario. Los quipus podrían haber contenido más información, como planes de construcción, patrones de danza e incluso aspectos de la historia inca. El quipu desmiente la noción de que las matemáticas florecen solo después de que una civilización ha desarrollado la escritura; sin embargo, las sociedades pueden alcanzar estados avanzados sin haber desarrollado registros escritos. Curiosamente, hoy en día existen sistemas informáticos cuyos gestores de archivos se llaman quipus, en honor a este antiguo dispositivo muy útil. \\

\noindent Una aplicación siniestra del quipu por los incas era como un calculador de muerte. Cuotas anuales de adultos y niños eran ritualmente sacrificadas, y esta empresa se planeaba utilizando un quipu. Algunos quipus representaban al imperio, y las cuerdas se referían a caminos y los nudos a víctimas sacrificiales.
