\noindent  A continuaci\'on se listan algunas leyes de \'algebra de conjuntos en los siguientes cuadros.\\

\begin{table}[ht]
    \centering
    \renewcommand{\arraystretch}{1.5} % Aumentar la altura de las filas de la tabla
    \begin{tabular}{|p{6cm}|p{5cm}|}
        \hline
        \textbf{Ley} & \textbf{Expresión} \\
        \hline
        Ley conmutativa de la unión  & $A \cup B =  B \cup A$ \\
        \hline
       Ley asociativa de la unión &  $A \cup (B \cap C) = (A \cup B)  \cup C$ \\
        \hline
        Ley conmutativa de la intersección & $A \cap B =  B \cap A$ \\
        \hline
        Ley asociativa de la intersección & $A \cap (B \cup C) = (A \cap B) \cap C$ \\
        \hline
        Ley de la Anulación & $A \cap \varnothing = \varnothing$ \hspace{1cm}  $A \cup U = U$ \\
        \hline
        
        
    \end{tabular}
    \caption{Leyes del Álgebra de Conjuntos (Parte 1).}
\end{table}

\begin{table}[ht]
        \centering
        \renewcommand{\arraystretch}{1.5} % Aumentar la altura de las filas de la tabla
        \begin{tabular}{|p{6cm}|p{5cm}|}
            \hline
            \textbf{Ley} & \textbf{Expresión} \\
            \hline
            Ley de la Complemento Universal & $A \cap U' = \varnothing$ \hspace{1cm} $A \cup \varnothing' = U$ \\
            \hline
            Primera ley distributiva & $A \cap (B \cup C) = (A \cap B) \cup (A \cap C)$ \\
            \hline

             Segunda ley distributiva &  $A \cup (B \cap C) = (A \cup B) \cap (A \cup C)$ \\
            \hline
            
            Ley de De Morgan & $(A \cap B)' = A' \cup B'$ \newline$(A \cup B)' = A' \cap B'$ \\
            \hline
            Ley del Complemento de la Intersección & $(A \cap B)' = A' \cup B'$ \\
            \hline
            Ley del Complemento de la Unión & $(A \cup B)' = A' \cap B'$ \\
            \hline
        \end{tabular}
        \caption{Leyes del Álgebra de Conjuntos (Parte 2).}
        \label{tabla:leyes-conjuntos}
    \end{table}